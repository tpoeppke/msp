%% LaTeX-Beamer template for KIT design
%% by Erik Burger, Christian Hammer
%% title picture by Klaus Krogmann
%%
%% version 2.1
%%
%% mostly compatible to KIT corporate design v2.0
%% http://intranet.kit.edu/gestaltungsrichtlinien.php
%%
%% Problems, bugs and comments to
%% burger@kit.edu

\documentclass[18pt]{beamer}

%% SLIDE FORMAT

% use 'beamerthemekit' for standard 4:3 ratio
% for widescreen slides (16:9), use 'beamerthemekitwide'

\usepackage{templates/beamerthemekit}
\usepackage{wasysym}
% \usepackage{templates/beamerthemekitwide}

%% TITLE PICTURE

% if a custom picture is to be used on the title page, copy it into the 'logos'
% directory, in the line below, replace 'mypicture' with the 
% filename (without extension) and uncomment the following line
% (picture proportions: 63 : 20 for standard, 169 : 40 for wide
% *.eps format if you use latex+dvips+ps2pdf, 
% *.jpg/*.png/*.pdf if you use pdflatex)

%\titleimage{mypicture}

%% TITLE LOGO

% for a custom logo on the front page, copy your file into the 'logos'
% directory, insert the filename in the line below and uncomment it

%\titlelogo{mylogo}

% (*.eps format if you use latex+dvips+ps2pdf,
% *.jpg/*.png/*.pdf if you use pdflatex)

%% TikZ INTEGRATION

% use these packages for PCM symbols and UML classes
% \usepackage{templates/tikzkit}
% \usepackage{templates/tikzuml}

% the presentation starts here

\title[Course Report]{Multilingual Speech Processing:\\ Context-Dependent Models}
\author{Jiacheng Yao, Tobias P\"oppke}

\institute{Cognitive Systems Lab}

% Bibliography

\usepackage[citestyle=authoryear,bibstyle=numeric,hyperref,backend=biber]{biblatex}
\addbibresource{templates/example.bib}
\bibhang1em

\begin{document}

% change the following line to "ngerman" for German style date and logos
\selectlanguage{english}

%title page
\begin{frame}
\titlepage
\end{frame}

%table of contents
%\begin{frame}{Outline/Gliederung}
%\tableofcontents
%\end{frame}

\begin{frame}{Initialization}
\begin{itemize}
\item Context-independent model already trained and available
\pause \item Load context-independent model from file
\pause \item Calculate paths for the training audio data
\pause \item Initialize the context-dependent model with these paths
\pause \item Build the context-clustering tree
\end{itemize}
\end{frame}

\begin{frame}{Initialization Parameters}
\begin{table}
\begin{tabular}{l | c | c | c}
Parameter & English 1 & English 2 & French \\ 
\hline \hline
Nr. of Gaussians per Model & 10 & -- & 10 \\
Clustering Context & 1 & -- & 2 \\
Max. number of leaves & 2000 & -- & 2000 \\
Min. number of samples & 500 & -- & 1000
\end{tabular}
\end{table}
\end{frame}


\begin{frame}{Training}
\begin{itemize}
\item Run iterations of training as in the context-independent case
\pause \item But only once for the complete context-dependent model
\pause \item Number of training iterations:
\begin{itemize}
\item French: 5
\item English: 5
\end{itemize}

\end{itemize}
\end{frame}

\begin{frame}{Decoding}
\begin{itemize}
\item The decoding process was the same as with the context-independent model
\pause \item Average word error rates for English Team:
\begin{itemize}
\item Development Set: 44\%
\item Recorded Data: 84\%
\end{itemize}
\end{itemize}
\end{frame}

\begin{frame}{Problems}
\begin{itemize}
\item Problem 1: Segmentation fault when starting to train the model
\pause \item Reason was that the models were not properly initialized
\pause \item Problem 2: No path could be found for decoding
\pause \item Cause is probably some error during the training
\end{itemize}
\end{frame}

\begin{frame}{Summary}
\begin{itemize}
\item It was interesting to train a real acoustic model
\pause \item Some problems occured because we didn't know what BioKIT was doing internally
\pause \item It would have been nice to implement some of the used algorithms
\end{itemize}
\end{frame}



\end{document}
